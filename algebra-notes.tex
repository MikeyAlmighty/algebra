\documentclass[11pt]{article}

% --- Packages ---
\usepackage[utf8]{inputenc}
\usepackage[T1]{fontenc}
\usepackage{amsmath, amssymb, amsfonts}
\usepackage{geometry}
\usepackage{graphicx}
\usepackage{enumitem}
\usepackage{fancyhdr}
\usepackage{color}
\usepackage{hyperref}
\usepackage{titlesec}
\usepackage{cancel} % For canceling terms
\usepackage{tikz}   % Diagrams and graphs
\usepackage{mathtools}
\usepackage{parskip}

% --- Page Setup ---
\geometry{margin=1in}
\pagestyle{fancy}
\fancyhf{}
\rhead{Algebra Notes}
\lhead{Michael Leibbrandt}
\rfoot{\thepage}

% --- Section Styling ---
\titleformat{\section}{\large\bfseries}{\thesection}{1em}{}
\titleformat{\subsection}{\normalsize\bfseries}{\thesubsection}{1em}{}

% --- Title ---
\title{\Huge \textbf{Algebra}\\\large Notes and Practice}
\author{Michael leibbrandt}
\date{\today}

% --- Document Begins ---
\begin{document}

\maketitle
\tableofcontents
\newpage

% ------------------------
\section{Introduction}

Algebra

This document contains concise notes, worked examples, and practice problems to help rebuild my algebra foundation.

% ------------------------
\section{Algebra Basics}

\subsection{Variables and Constants}

A \textbf{variable} is a symbol (often a letter) that represents a number. A \textbf{constant} is a fixed value.

Example:
\[
x + 3 = 7 \quad \Rightarrow \quad x = 4
\]

\subsection{Order of Operations}

Remember PEMDAS:
\begin{itemize}
  \item \textbf{P}arentheses
  \item \textbf{E}xponents
  \item \textbf{MD} Multiplication and Division (left to right)
  \item \textbf{AS} Addition and Subtraction (left to right)
\end{itemize}

\[
2 + 3 \times (4^2 - 1) = 2 + 3 \times (16 - 1) = 2 + 3 \times 15 = 47
\]

% ------------------------
\section{Linear Equations}

\subsection{Solving One-Step Equations}

Solve:
\[
x - 5 = 9 \quad \Rightarrow \quad x = 14
\]

% ------------------------
% Continue building sections here...

\end{document}
