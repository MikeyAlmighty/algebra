\documentclass[11pt]{article}

% --- Packages ---
\usepackage[utf8]{inputenc}
\usepackage[T1]{fontenc}
\usepackage{amsmath, amssymb, amsfonts}
\usepackage{geometry}
\usepackage{graphicx}
\usepackage{enumitem}
\usepackage{fancyhdr}
\usepackage{color}
\usepackage{hyperref}
\usepackage[most]{tcolorbox}
\usepackage{titlesec}
\usepackage{cancel} % For canceling terms
\usepackage{tikz}   % Diagrams and graphs
\usepackage{mathtools}
\usepackage{parskip}

% --- Page Setup ---
\geometry{margin=1in}
\pagestyle{fancy}
\fancyhf{}
\rhead{Algebra}
\lhead{Michael}
\rfoot{\thepage}

% --- Section Styling ---
\titleformat{\section}{\large\bfseries}{\thesection}{1em}{}
\titleformat{\subsection}{\normalsize\bfseries}{\thesubsection}{1em}{}

% --- Title ---
\title{\Huge \textbf{Algebra}\\\large Notes and Practice}
\author{Michael Leibbrandt}
\date{\today}

% --- Document Begins ---
\begin{document}

\maketitle
\tableofcontents
\newpage

% ------------------------
\section{Introduction}

This document contains concise notes and worked examples.

% ------------------------
\section{Algebra Basics}

\subsection{Constants}

A \textbf{constant} is a fixed value that does not change. Unlike a \textbf{variable}, which can represent different numbers, a \textbf{constant} always has the same value.

Examples of \textbf{constant} include:

\begin{itemize}
  \item Specific numbers like \( 2 \), \( -7 \), or \( \frac{1}{2} \)
  \item Mathematical constants like \( \pi \approx 3.1416 \) or \( e \approx 2.718 \)
\end{itemize}

In the equation:
\[
x + 3 = 7
\]
the number \( 3 \) and \( 7 \) are constants — they stay the same, while \( x \) is the variable we solve for.

\subsection{Variables}

A \textbf{variable} is a symbol, usually a letter like \( x \), \( y \), or \( z \), that represents an unknown or changeable value.

For example, in the equation:
\[
x + 3 = 7
\]
the variable \( x \) represents a number. Solving the equation means finding the value of \( x \) that makes the equation true in this case, \( x = 4 \).

\textbf{Variables} are fundamental in algebra because they allow us to generalize problems and create formulas.
\subsection{Coefficients}

A \textbf{coefficient} is the numerical factor multiplied by a variable in an algebraic expression.

For example, in the term:
\[
5x
\]
the number \( 5 \) is the coefficient of the variable \( x \). It tells us how many times \( x \) is being counted or scaled.

More examples:
\begin{itemize}
  \item In \( -3y \), the coefficient is \( -3 \)
  \item In \( \frac{1}{2}a \), the coefficient is \( \frac{1}{2} \)
  \item In \( z \), the coefficient is implicitly \( 1 \), since \( z = 1 \cdot z \)
\end{itemize}

Coefficients help determine the slope of a line in linear equations and play a major role in simplifying and solving expressions.
\section{Equations vs. Expressions}

Understanding the difference between expressions and equations is essential in algebra.
\subsection{Expressions}

An \textbf{expression} is a combination of numbers, variables, and operations (like addition or multiplication), but it does \textbf{not} contain an equals sign.

Examples:
\[
2x + 5,\quad 3a^2 - 4,\quad \frac{1}{2}y
\]

Expressions represent a value, but not a complete statement to solve. You can simplify or evaluate expressions, but you cannot "solve" them unless they're part of an equation.

\subsection{Equations}

An \textbf{equation} is a mathematical statement that two expressions are equal. It always contains an equals sign (\( = \)) and usually involves finding the value of a variable that makes the equation true.

Examples:
\[
2x + 5 = 11, \quad a^2 = 16, \quad \frac{1}{2}y = 3
\]

Solving an \textbf{equation} means determining the value(s) of the variable(s) that make both sides equal.

\subsection*{Summary Table}

\begin{center}
\begin{tabular}{|c|c|}
\hline
\textbf{Expression} & \textbf{Equation} \\
\hline
No equals sign & Has an equals sign \\
Represents a value & Represents a relationship \\
Can be simplified or evaluated & Can be solved \\
Example: \( 3x + 2 \) & Example: \( 3x + 2 = 11 \) \\
\hline
\end{tabular}
\end{center}

\section{The Associative Property}

The \textbf{associative property} refers to the grouping of terms using parentheses in addition or multiplication. It tells us that the way numbers are grouped does not change the result — only the order of operations inside parentheses changes, not the outcome.

\subsection{Associative Property of Addition}

The associative property of addition states:

\[
(a + b) + c = a + (b + c)
\]

You can add numbers in any grouping, and the sum will stay the same.

\textbf{Example:}
\[
(2 + 3) + 4 = 5 + 4 = 9
\]
\[
2 + (3 + 4) = 2 + 7 = 9
\]

So, \( (2 + 3) + 4 = 2 + (3 + 4) \).

\subsection{Associative Property of Multiplication}

The associative property of multiplication states:

\[
(a \times b) \times c = a \times (b \times c)
\]

You can multiply in any grouping, and the product remains unchanged.

\textbf{Example:}
\[
(2 \times 3) \times 4 = 6 \times 4 = 24
\]
\[
2 \times (3 \times 4) = 2 \times 12 = 24
\]

So, \( (2 \times 3) \times 4 = 2 \times (3 \times 4) \).

\section{The Commutative Property}

The \textbf{commutative property} describes how the order of numbers does not affect the result when adding or multiplying. It applies only to \textbf{addition and multiplication} — not subtraction or division.

\subsection{Commutative Property of Addition}

The commutative property of addition states:

\[
a + b = b + a
\]

You can change the order of the numbers being added without changing the sum.

\textbf{Example:}
\[
4 + 7 = 11 \quad \text{and} \quad 7 + 4 = 11
\]

So, \( 4 + 7 = 7 + 4 \)

\subsection{Commutative Property of Multiplication}

The commutative property of multiplication states:

\[
a \times b = b \times a
\]

You can change the order of the numbers being multiplied without changing the product.

\textbf{Example:}
\[
6 \times 5 = 30 \quad \text{and} \quad 5 \times 6 = 30
\]

So, \( 6 \times 5 = 5 \times 6 \)
\section{Like Terms}

\textbf{Like terms} are terms that have the same variable(s) raised to the same power(s). Only the numerical coefficients can be different. Like terms can be combined using addition or subtraction.

\subsection{Addition and Subtraction of Like Terms}

To combine like terms, simply add or subtract their coefficients.

\textbf{Example 1:}
\[
3x + 5x = (3 + 5)x = 8x
\]

\textbf{Example 2:}
\[
7a^2 - 2a^2 = 5a^2
\]

\textbf{Note:} You \textit{cannot} combine terms that are not like terms.

\[
4x + 2x^2 \neq 6x^2
\quad \text{(not like terms)}
\]

\subsection{Multiplication and Division of Like Terms}

When multiplying or dividing like terms, you combine coefficients and apply exponent rules to the variables.

\textbf{Multiplication Example:}
\[
(3x)(2x) = 6x^2
\]

\[
(4a^2)(-2a^3) = -8a^5
\]

\textbf{Division Example:}
\[
\frac{10x^3}{2x} = 5x^2
\]

\[
\frac{-6y^4}{3y^2} = -2y^2
\]

\subsection*{Key Idea}

\begin{itemize}
  \item \textbf{Addition/Subtraction:} Combine only like terms (same variables and exponents)
  \item \textbf{Multiplication/Division:} Use exponent rules, even for unlike terms
\end{itemize}
\section{The Distributive Property}

The \textbf{distributive property} connects multiplication and addition or subtraction. It allows you to multiply a number or variable by each term inside parentheses.

\[
a(b + c) = ab + ac
\quad \text{and} \quad
a(b - c) = ab - ac
\]

This property is used frequently in algebra to expand expressions and solve equations.

\subsection*{Examples}

\textbf{Example 1 (with numbers):}
\[
3(4 + 5) = 3 \cdot 9 = 27
\]
\[
3 \cdot 4 + 3 \cdot 5 = 12 + 15 = 27
\]

\textbf{Example 2 (with variables):}
\[
x(2 + y) = 2x + xy
\]

\textbf{Example 3 (with subtraction):}
\[
5(a - 3) = 5a - 15
\]

\subsection*{Why It Matters}

The distributive property helps you:
\begin{itemize}
  \item Expand expressions like \( 2(x + 3) \)
  \item Simplify algebraic expressions
  \item Solve equations more efficiently
  \item Factor expressions in reverse
\end{itemize}
\subsection*{More on the Distributive Property}

The distributive property is also essential when working with variables, negative numbers, and factoring expressions.

\subsubsection*{Example with Variables and Negatives}

\[
-2(x - 4) = -2 \cdot x + (-2) \cdot (-4) = -2x + 8
\]

Notice:
\begin{itemize}
  \item The negative sign distributes to both terms
  \item Be careful with signs: \( -2 \cdot -4 = +8 \)
\end{itemize}

\subsubsection*{Common Mistake to Avoid}

Incorrect:
\[
3(x + 2) = 3x + 2 \quad \text{(Only distributed to } x \text{)}
\]

Correct:
\[
3(x + 2) = 3x + 6 \quad
\]

Always distribute to \textit{every} term inside the parentheses.

\subsubsection*{Using the Distributive Property to Factor}

The distributive property also works \textit{in reverse}, which is how we factor expressions.

\[
6x + 12 = 6(x + 2)
\]

Here, we pulled out the common factor of 6 — essentially undoing the distribution.

Factoring is the process of writing an expression as a product using the distributive property in reverse.

\textbf{Tip:} Look for a greatest common factor (GCF) before factoring!
\section{Polynomials}

A \textbf{polynomial} is an expression made up of variables, constants, and exponents, combined using addition, subtraction, and multiplication — but no variables in the denominator or under radicals.

\subsection*{Examples of Polynomials}

\[
3x^2 + 2x - 5, \quad x^3 - 4x + 7, \quad 2a^2b + 3ab^2
\]

---

\subsection{Addition and Subtraction of Polynomials}

To add or subtract polynomials:
\begin{itemize}
  \item Combine like terms (same variables raised to the same powers)
  \item Add/subtract the coefficients of like terms
\end{itemize}

\textbf{Example 1 — Addition:}
\[
(2x^2 + 3x + 1) + (x^2 + 4x - 5)
\]
\[
= (2x^2 + x^2) + (3x + 4x) + (1 - 5) = 3x^2 + 7x - 4
\]

\textbf{Example 2 — Subtraction:}
\[
(5x^2 - 2x + 6) - (3x^2 + x - 4)
\]
\[
= (5x^2 - 3x^2) + (-2x - x) + (6 + 4) = 2x^2 - 3x + 10
\]

---

\subsection{Multiplication of Polynomials}

To multiply polynomials:
\begin{itemize}
  \item Use the distributive property (FOIL for binomials)
  \item Multiply each term in the first polynomial by each term in the second
  \item Combine like terms
\end{itemize}

\textbf{Example:}
\[
(x + 2)(x + 5)
= x(x + 5) + 2(x + 5)
= x^2 + 5x + 2x + 10
= x^2 + 7x + 10
\]

\textbf{Another Example:}
\[
(2x - 3)(x^2 + x - 4)
= 2x(x^2 + x - 4) - 3(x^2 + x - 4)
\]
\[
= 2x^3 + 2x^2 - 8x - 3x^2 - 3x + 12
= 2x^3 - x^2 - 11x + 12
\]

---

\subsection{Division of Polynomials (Intro)}

Dividing polynomials can be done using:
\begin{itemize}
  \item Long division
  \item Synthetic division (when dividing by linear terms like \( x - a \))
\end{itemize}

\textbf{Basic Example:}
\[
\frac{6x^2 + 9x}{3x} = \frac{6x^2}{3x} + \frac{9x}{3x} = 2x + 3
\]

\textbf{Note:} More complex division techniques will be covered in a later section.
\begin{tcolorbox}[title=Polynomial Operations Summary, colback=blue!5!white, colframe=blue!75!black, sharp corners=southwest, boxrule=0.8pt]

\textbf{Polynomials} are expressions with variables and constants using only addition, subtraction, and multiplication (no variables in denominators or exponents).

\vspace{1ex}
\textbf{Addition/Subtraction:}
\begin{itemize}
  \item Combine like terms (same variable and exponent)
  \item Only coefficients are added or subtracted
\end{itemize}

\textbf{Multiplication:}
\begin{itemize}
  \item Use the distributive property or FOIL
  \item Multiply each term in one polynomial by each term in the other
  \item Combine like terms
\end{itemize}

\textbf{Division:}
\begin{itemize}
  \item Simplify each term if possible
  \item For complex division, use long division or synthetic division (covered later)
\end{itemize}

\end{tcolorbox}

\subsection{Long Division}

Long division is a step-by-step method for dividing numbers or algebraic expressions.

\begin{itemize}
  \item \textbf{Dividend:} the number or expression being divided
  \item \textbf{Divisor:} the number or expression you are dividing by
  \item \textbf{Quotient:} the result of the division (goes on top)
\end{itemize}

---

\subsubsection*{Example 1: Long Division with Whole Numbers}

Divide:
\[
125 \div 5
\]

Here:
\begin{itemize}
  \item Dividend = 125
  \item Divisor = 5
  \item Quotient = 25
\end{itemize}

\[
\begin{array}{r|l}
5\, &\,125 \\
\hline
& 25
\end{array}
\]

Steps:
\begin{enumerate}
  \item Divide 12 by 5 → 2 (since \( 5 \times 2 = 10 \))
  \item Subtract: \( 12 - 10 = 2 \), bring down the 5 → 25
  \item Divide 25 by 5 → 5 (since \( 5 \times 5 = 25 \))
  \item Final result: \( 25 \)
\end{enumerate}

---

\subsubsection*{Example 2: Polynomial Long Division}

Divide:
\[
\frac{x^2 + 3x + 2}{x + 1}
\]

Here:
\begin{itemize}
  \item Dividend = \( x^2 + 3x + 2 \)
  \item Divisor = \( x + 1 \)
  \item Quotient = the expression that results from division
\end{itemize}

\textbf{Step 1:} Divide leading terms:
\[
x^2 \div x = x
\]

\textbf{Step 2:} Multiply and subtract:
\[
(x^2 + 3x + 2) - (x)(x + 1) = x^2 + 3x + 2 - (x^2 + x) = 2x + 2
\]

\textbf{Step 3:} Divide leading terms again:
\[
2x \div x = 2
\]

\textbf{Step 4:} Multiply and subtract:
\[
(2x + 2) - 2(x + 1) = 2x + 2 - (2x + 2) = 0
\]

So the division is exact.

\textbf{Final Answer:}
\[
\frac{x^2 + 3x + 2}{x + 1} = x + 2
\]

---

\begin{tcolorbox}[title=Terminology Review, colback=blue!5!white, colframe=blue!75!black]
\begin{itemize}
  \item \textbf{Dividend:} what you're dividing \quad (e.g., \( x^2 + 3x + 2 \))
  \item \textbf{Divisor:} what you're dividing by \quad (e.g., \( x + 1 \))
  \item \textbf{Quotient:} result of the division \quad (e.g., \( x + 2 \))
\end{itemize}
\end{tcolorbox}
\subsection{Multivariable Polynomial Long Division}

Polynomial long division can also be performed with expressions that include more than one variable. The process is similar, but care must be taken to match like terms correctly and order terms consistently by degree.

---

\subsubsection*{Example: Divide \( 6x^2y + 9xy^2 \) by \( 3xy \)}

Here:
\begin{itemize}
  \item Dividend: \( 6x^2y + 9xy^2 \)
  \item Divisor: \( 3xy \)
\end{itemize}

\textbf{Step 1: Divide the first term of the dividend by the first term of the divisor}
\[
\frac{6x^2y}{3xy} = 2x
\]

\textbf{Step 2: Multiply the entire divisor by \( 2x \)}
\[
2x \cdot (3xy) = 6x^2y
\]

\textbf{Step 3: Subtract}
\[
(6x^2y + 9xy^2) - 6x^2y = 9xy^2
\]

\textbf{Step 4: Divide next term}
\[
\frac{9xy^2}{3xy} = 3y
\]

\textbf{Step 5: Multiply and subtract}
\[
3y \cdot (3xy) = 9xy^2
\]
\[
9xy^2 - 9xy^2 = 0
\]

\textbf{Final Answer:}
\[
\frac{6x^2y + 9xy^2}{3xy} = 2x + 3y
\]

---

\begin{tcolorbox}[title=Tips for Multivariable Long Division, colback=yellow!5!white, colframe=yellow!80!black]
\begin{itemize}
  \item Organize terms in descending order of one variable (typically \( x \))
  \item Divide one term at a time, matching both variable parts and coefficients
  \item Use standard subtraction to cancel each step before proceeding
\end{itemize}
\end{tcolorbox}

\end{document}
