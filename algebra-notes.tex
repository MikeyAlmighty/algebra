\documentclass[11pt]{article}

% --- Packages ---
\usepackage[utf8]{inputenc}
\usepackage[T1]{fontenc}
\usepackage{amsmath, amssymb, amsfonts}
\usepackage{geometry}
\usepackage{graphicx}
\usepackage{enumitem}
\usepackage{fancyhdr}
\usepackage{color}
\usepackage{hyperref}
\usepackage{titlesec}
\usepackage{cancel} % For canceling terms
\usepackage{tikz}   % Diagrams and graphs
\usepackage{mathtools}
\usepackage{parskip}

% --- Page Setup ---
\geometry{margin=1in}
\pagestyle{fancy}
\fancyhf{}
\rhead{Algebra Notes}
\lhead{Michael Leibbrandt}
\rfoot{\thepage}

% --- Section Styling ---
\titleformat{\section}{\large\bfseries}{\thesection}{1em}{}
\titleformat{\subsection}{\normalsize\bfseries}{\thesubsection}{1em}{}

% --- Title ---
\title{\Huge \textbf{Algebra}\\\large Notes and Practice}
\author{Michael leibbrandt}
\date{\today}

% --- Document Begins ---
\begin{document}

\maketitle
\tableofcontents
\newpage

% ------------------------
\section{Introduction}

This document contains concise notes, worked examples, and practice problems to help rebuild my algebra foundation.

% ------------------------
\section{Algebra Basics}

\subsection{Constants}

A \textbf{constant} is a fixed value that does not change. Unlike a \textbf{variable}, which can represent different numbers, a \textbf{constant} always has the same value.

Examples of \textbf{constant} include:

\begin{itemize}
  \item Specific numbers like \( 2 \), \( -7 \), or \( \frac{1}{2} \)
  \item Mathematical constants like \( \pi \approx 3.1416 \) or \( e \approx 2.718 \)
\end{itemize}

In the equation:
\[
x + 3 = 7
\]
the number \( 3 \) and \( 7 \) are constants — they stay the same, while \( x \) is the variable we solve for.

\subsection{Variables}

A \textbf{variable} is a symbol, usually a letter like \( x \), \( y \), or \( z \), that represents an unknown or changeable value.

For example, in the equation:
\[
x + 3 = 7
\]
the variable \( x \) represents a number. Solving the equation means finding the value of \( x \) that makes the equation true in this case, \( x = 4 \).

\textbf{Variables} are fundamental in algebra because they allow us to generalize problems and create formulas.
\subsection{Coefficients}

A \textbf{coefficient} is the numerical factor multiplied by a variable in an algebraic expression.

For example, in the term:
\[
5x
\]
the number \( 5 \) is the coefficient of the variable \( x \). It tells us how many times \( x \) is being counted or scaled.

More examples:
\begin{itemize}
  \item In \( -3y \), the coefficient is \( -3 \)
  \item In \( \frac{1}{2}a \), the coefficient is \( \frac{1}{2} \)
  \item In \( z \), the coefficient is implicitly \( 1 \), since \( z = 1 \cdot z \)
\end{itemize}

Coefficients help determine the slope of a line in linear equations and play a major role in simplifying and solving expressions.
\section{Equations vs. Expressions}

Understanding the difference between expressions and equations is essential in algebra.
\subsection{Expressions}

An \textbf{expression} is a combination of numbers, variables, and operations (like addition or multiplication), but it does \textbf{not} contain an equals sign.

Examples:
\[
2x + 5,\quad 3a^2 - 4,\quad \frac{1}{2}y
\]

Expressions represent a value, but not a complete statement to solve. You can simplify or evaluate expressions, but you cannot "solve" them unless they're part of an equation.

\subsection{Equations}

An \textbf{equation} is a mathematical statement that two expressions are equal. It always contains an equals sign (\( = \)) and usually involves finding the value of a variable that makes the equation true.

Examples:
\[
2x + 5 = 11, \quad a^2 = 16, \quad \frac{1}{2}y = 3
\]

Solving an \textbf{equation} means determining the value(s) of the variable(s) that make both sides equal.

\subsection*{Summary Table}

\begin{center}
\begin{tabular}{|c|c|}
\hline
\textbf{Expression} & \textbf{Equation} \\
\hline
No equals sign & Has an equals sign \\
Represents a value & Represents a relationship \\
Can be simplified or evaluated & Can be solved \\
Example: \( 3x + 2 \) & Example: \( 3x + 2 = 11 \) \\
\hline
\end{tabular}
\end{center}

\section{The Associative Property}

The \textbf{associative property} refers to the grouping of terms using parentheses in addition or multiplication. It tells us that the way numbers are grouped does not change the result — only the order of operations inside parentheses changes, not the outcome.

\subsection{Associative Property of Addition}

The associative property of addition states:

\[
(a + b) + c = a + (b + c)
\]

You can add numbers in any grouping, and the sum will stay the same.

\textbf{Example:}
\[
(2 + 3) + 4 = 5 + 4 = 9
\]
\[
2 + (3 + 4) = 2 + 7 = 9
\]

So, \( (2 + 3) + 4 = 2 + (3 + 4) \).

\subsection{Associative Property of Multiplication}

The associative property of multiplication states:

\[
(a \times b) \times c = a \times (b \times c)
\]

You can multiply in any grouping, and the product remains unchanged.

\textbf{Example:}
\[
(2 \times 3) \times 4 = 6 \times 4 = 24
\]
\[
2 \times (3 \times 4) = 2 \times 12 = 24
\]

So, \( (2 \times 3) \times 4 = 2 \times (3 \times 4) \).

\section{The Commutative Property}

The \textbf{commutative property} describes how the order of numbers does not affect the result when adding or multiplying. It applies only to \textbf{addition and multiplication} — not subtraction or division.

\subsection{Commutative Property of Addition}

The commutative property of addition states:

\[
a + b = b + a
\]

You can change the order of the numbers being added without changing the sum.

\textbf{Example:}
\[
4 + 7 = 11 \quad \text{and} \quad 7 + 4 = 11
\]

So, \( 4 + 7 = 7 + 4 \)

\subsection{Commutative Property of Multiplication}

The commutative property of multiplication states:

\[
a \times b = b \times a
\]

You can change the order of the numbers being multiplied without changing the product.

\textbf{Example:}
\[
6 \times 5 = 30 \quad \text{and} \quad 5 \times 6 = 30
\]

So, \( 6 \times 5 = 5 \times 6 \)
\section{The Distributive Property}

The \textbf{distributive property} connects multiplication and addition (or subtraction). It states that multiplying a number by a sum is the same as multiplying it by each term separately and then adding the results.

\[
a(b + c) = ab + ac
\]
\[
a(b - c) = ab - ac
\]

This property is especially useful for simplifying expressions and solving equations.

\subsection*{Examples}

\textbf{Example 1:}
\[
3(4 + 5) = 3 \times 9 = 27
\]
\[
3 \times 4 + 3 \times 5 = 12 + 15 = 27
\]

\textbf{Example 2 (with variables):}
\[
x(2 + y) = x \cdot 2 + x \cdot y = 2x + xy
\]

\textbf{Example 3 (with subtraction):}
\[
5(a - 3) = 5a - 15
\]

\subsection*{Why It's Important}

The distributive property is essential when:
\begin{itemize}
  \item Expanding expressions like \( 2(x + 3) \)
  \item Factoring expressions like \( 3x + 6 \)
  \item Solving equations efficiently
\end{itemize}
\section{Like Terms}

\textbf{Like terms} are terms that have the same variable(s) raised to the same power(s). Only the numerical coefficients can be different. Like terms can be combined using addition or subtraction.

\subsection{Addition and Subtraction of Like Terms}

To combine like terms, simply add or subtract their coefficients.

\textbf{Example 1:}
\[
3x + 5x = (3 + 5)x = 8x
\]

\textbf{Example 2:}
\[
7a^2 - 2a^2 = 5a^2
\]

\textbf{Note:} You \textit{cannot} combine terms that are not like terms.

\[
4x + 2x^2 \neq 6x^2
\quad \text{(not like terms)}
\]

\subsection{Multiplication and Division of Like Terms}

When multiplying or dividing like terms, you combine coefficients and apply exponent rules to the variables.

\textbf{Multiplication Example:}
\[
(3x)(2x) = 6x^2
\]

\[
(4a^2)(-2a^3) = -8a^5
\]

\textbf{Division Example:}
\[
\frac{10x^3}{2x} = 5x^2
\]

\[
\frac{-6y^4}{3y^2} = -2y^2
\]

\subsection*{Key Idea}

\begin{itemize}
  \item \textbf{Addition/Subtraction:} Combine only like terms (same variables and exponents)
  \item \textbf{Multiplication/Division:} Use exponent rules, even for unlike terms
\end{itemize}

\end{document}
